% !TeX spellcheck = ru_RU
\documentclass[a4paper, 14pt, unknownkeysallowed]{extreport}
\include{../common/settings}

\begin{document}
\include{title}
\setcounter{page}{2}

\chapter*{Введение}
\addcontentsline{toc}{chapter}{Введение}
Цель работы: приобрести навыки работы в системе Common Lisp.

Задачи работы: изучить работу форм — функций, которые особым образом обрабатывают свои аргументы и особенности их работы в Lisp.

\chapter{Практические задания}

\section{Написать функцию, которая принимает целое число и возвращает первое четное число, не меньшее аргумента}
\begin{lstlisting}
(defun f1 (x)
	(+ x (mod x 2))
)
\end{lstlisting}

\section{Написать функцию, которая принимает число и возвращает число того же знака, но с модулем на 1 больше модуля аргумента}

\begin{lstlisting}
(defun f2 (x)
	(if (minusp x) (- x 1) (+ x 1))
)
\end{lstlisting}

\section{Написать функцию, которая принимает два числа и возвращает список из этих чисел, расположенный по возрастанию}

\begin{lstlisting}
(defun f3 (a b)
	(if (< a b) (list a b) (list b a))
)
\end{lstlisting}

\section{Написать функцию, которая принимает три числа и возвращает Т только тогда, когда первое число расположено между вторым и третьим}

\begin{lstlisting}
(defun f4 (a b c)
	(if 
		(or
			(and (< b a) (< a c))
			(and (> b a) (> a c))
		)
		T
		Nil
	)
)
\end{lstlisting}

\section{Каков результат вычисления следующих выражений}

\begin{lstlisting}
(and 'fee 'fie 'foe) ;foe
(or 'fee 'fie 'foe) ;fee
(or nil 'fie 'foe) ;fie
(and nil 'fie 'foe) ;nil
(and (equal 'abc 'abc) 'yes) ;yes
(or (equal 'abc 'abc) 'yes) ;T
\end{lstlisting}


\section{Написать предикат, который принимает два числа-аргумента и возвращает Т, если первое число не меньше второго}.

\begin{lstlisting}
(defun f6 (a b) 
	(>= a b)
)
\end{lstlisting} 

\section{Какой из следующих двух вариантов предиката ошибочен и почему?}

\begin{lstlisting}
(defun pred1 (x)
	(and (numberp x) (plusp x)))

(defun pred2 (x)
	(and (plusp x)(numberp x)))
\end{lstlisting} 

Ошибочен второй вариант предиката, так как в ходе его вычисления сначала будет вычислено (plusp x), что может привести к ошибке, если x не численного типа.

\section{Решить задачу 4, используя для ее решения конструкции: только IF, только COND, только AND/OR.}
Написать функцию, которая принимает три числа и возвращает Т только тогда, когда первое число расположено между вторым и третьим.
\begin{lstlisting}
(defun f81 (a b c )
	(if (< b a)
		(if (< a c)
		T
		Nil)
		Nil))
\end{lstlisting} 

\begin{lstlisting}
(defun f82 (a b c )
	(cond ((< b a) (cond ((< a c) T) (T Nil))) 
			(T Nil))
)
\end{lstlisting} 

\begin{lstlisting}
(defun f83 (a b c )
	(and (< b a) (< a c)))
\end{lstlisting} 

\section{Переписать функцию how-alike, приведенную в лекции и использующую COND, используя только конструкции IF, AND/OR.}

\begin{lstlisting}
(defun how_alike (x y)
	(cond ((or (= x y) (equal x y)) 'the_same)
		((and (oddp x) (oddp y)) 'both_odd)
		((and (evenp x) (evenp y)) 'both_even)
		(T 'difference)
))
\end{lstlisting} 

\begin{lstlisting}
(defun how_alike (x y)
	(if (if (= x y)
		(equal x y))
		'the_same
		(if (if (oddp x)
			(oddp y))
		'both_odd
		(if (if (evenp x)
			(evenp y))
		'both_even
		'difference
))))
\end{lstlisting} 

\begin{lstlisting}
(defun how_alike (x y)
	(or
		(and (= x y) (equal x y) 'the_same)
		(and (oddp x) (oddp y) 'both_odd)
		(and (evenp x) (evenp y) 'both_even)
		'difference
))
\end{lstlisting} 

\end{document}