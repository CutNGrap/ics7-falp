% !TeX spellcheck = ru_RU
\newpage
\chapter*{Введение}
\addcontentsline{toc}{chapter}{Введение}
Матричная алгебра имеет обширные применения в различных отраслях знания – в математике, физике, информатике, экономике. Например, матрицы используется для решения систем алгебраических и дифференциальных уравнений, нахождения значений физических величин в квантовой теории, шифрования сообщений в Интернете.

Важной стороной работы с матрицами в программировании является оптимизация матричных операций (умножение, сложение, транспозиция и так далее), так как во многих задачах размеры матриц могут достигать больших значений. 
В данной работе пойдет речь об оптимизации операции умножения матриц на языке Common Lisp.


\textbf{Целью данной работы} является реализация и сравнительный анализ функций перемножения матриц с использованием хвостовой рекурсии и функционалов.
