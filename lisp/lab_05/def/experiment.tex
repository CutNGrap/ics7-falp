% !TeX spellcheck = ru_RU
\chapter{Исследовательская часть}


Ниже приведены технические характеристики устройства, на котором было проведено измерение времени работы ПО:

\begin{itemize}
	\item операционная система Ubuntu 22.04;
	\item оперативная память 8 Гб 2133 МГц;
	\item процессор Intel Core i5-8300H с тактовой частотой 2.30 ГГц.
\end{itemize}

\section{Время выполнения реализаций алгоритмов}

Для замеров времени использовалась функция замера процессорного времени get-intrnal-run-time языка Common Lisp.

Функция используется дважды --- в начале и в конце замера времени, затем значения начальное значение вычитается из конечного.

Замеры проводились для квадратных матриц размерами {100, 200,...,600}.
Матрицы заполнялись случайными числами от 0 до 1000.

В таблице \ref{tbl:even} представлены замеры времени работы для каждого из алгоритмов.

\begin{table}[h]
	\centering
	\caption{Результаты замеров времени работы реализаций алгоритмов для различных размерностей матриц (в мс)}
	\label{tbl:even}
	\begin{tabular}{|c| c |c|}
	\hline
	Размерность & Рекурсия & Функционалы\\
	\hline
	100 & 67,7 &16,2\\
	\hline
	200 & 1025,2 &124,9\\
	\hline
	300 & 4905,7&470,4\\
	\hline
	400 & 14990,9&1243,5\\
	\hline
	500 & 36474,5&2428,4\\
	\hline
	600 & 76315,9&4572,5\\
	\hline
	
	\end{tabular}
	
\end{table}

\clearpage

На рисунке \ref{img:fig} приведена графическая интерпретация результатов измерения.
\img{0.7}{fig}{Графическая интерпретация результатов измерения}



Из полученных результатов можно сделать вывод, что реализация алгоритма с использованием функционалов работает быстрее рекурсивной реализации.

\section*{Вывод}

В результате экспериментов было выявлено, что реализация алгоритма умножения матриц с использованием функционалов работает быстрее рекурсивной реализации. Для квадратных матриц размерностью $600\times600$ реализации дают следующие результаты:
\begin{itemize}
	\item рекурсивный алгоритм --- 76,3 с;
	\item алгоритм с использованием функционалов --- 4,5 с;
\end{itemize}

Таким образом, использование функционалов является позволяет ускорить выполнение алгоритма умножения матриц. Для матриц размерностью $600\times600$ рекурсивная реализация проигрывает более чем в 16 раз.

