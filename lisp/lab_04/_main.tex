% !TeX spellcheck = ru_RU
\documentclass[a4paper, 14pt, unknownkeysallowed]{extreport}
\include{../common/settings}

\begin{document}
\include{title}
\setcounter{page}{2}

\chapter*{Введение}
\addcontentsline{toc}{chapter}{Введение}
Цель работы: приобрести навыки работы с управляющими структурами Lisp.

Задачи работы: изучить работу функций с произвольным количеством аргументов, функций
разрушающих и неразрушающих структуру исходных аргументов.

\chapter{Практические задания}

\section{Чем принципиально отличаются функции cons, list, append?
	Пусть (setf lst1 '( a b c)) 
	(setf lst2 '( d e)). 
	Каковы результаты вычисления следующих выражений?}
\begin{lstlisting}
(cons lstl lst2) ;((a b c) d e)
(list lst1 lst2) ;((a b c) (d e))
(append lst1 lst2) ; (a b c d e)
\end{lstlisting}

\section{Каковы результаты вычисления следующих выражений, и почему?}

\begin{lstlisting}
(reverse '(a b c)) ;(c b a)
(reverse ()) ;Nil
(reverse '(a b (c (d)))) ;((c (d)) b a)
(reverse '((a b c))) ;((a b c))
(reverse '(a)) ;(a)
(last '(a b c)) ;(c)
(last '(a b (c))) ;((c))
(last '(a)) ;(a)
(last ()) ;Nil
(last '((a b c))) ;((a b c))
\end{lstlisting}

\section{Написать, по крайней мере, два варианта функции, которая возвращает
	последний элемент своего списка-аргумента.}

\begin{lstlisting}
(defun f3 (x)
	(car (last x))
)
\end{lstlisting}

\begin{lstlisting}
(defun f32 (lst)
	(cond
		((cdr  lst) (f32  (cdr lst)))
		(T(car lst))
	)
)
\end{lstlisting}

\section{Написать, по крайней мере, два варианта функции, которая возвращает
	свой список аргумент без последнего элемента. }

\begin{lstlisting}
(defun without_last (x)
	(
		cond 
		(
			(cdr x) 
			(cons (car x) (f4 (cdr x)))
		)
		(
			T Nil
		)
	)
)
\end{lstlisting}

\begin{lstlisting}
(defun f4 (x)
	(reverse (cdr (reverse x)))
)
\end{lstlisting}

\section{Напишите функцию swap-first-last, которая переставляет в списке аргументе первый и последний элементы}

\begin{lstlisting}
(defun swap-first-last (x)
	(append 
		(last x)
		(without_last (cdr x))
		(cons (car x) Nil)
	)
)
\end{lstlisting}


\section{}
Написать простой вариант игры в кости, в котором бросаются две
правильные кости. Если сумма выпавших очков равна 7 или 11 —
выигрыш, если выпало (1,1) или (6,6) — игрок имеет право снова
бросить кости, во всех остальных случаях ход переходит ко второму
игроку, но запоминается сумма выпавших очков. Если второй игрок не
выигрывает абсолютно, то выигрывает тот игрок, у которого больше
очков. Результат игры и значения выпавших костей выводить на экран с помощью функции print. 

\begin{lstlisting}
(defvar dices)
(defvar roll1)
(defvar roll2)
(defvar result)

(defun sum (throw)
	(+ (car throw) (cdr throw)))

(defun 1_roll ()
	(+ (random 6) 1))
	
(defun roll ()
	(cons (1_roll) (1_roll))
)

(defun reroll_d (sum_count)
	(if (or (= sum_count 2) (= sum_count 12))
		T
		Nil))
		
(defun win_d (sum_count)
	(if (or (= sum_count 7) (= sum_count 11))
	T
	Nil))
	
(defun turn ()
	(setq dices (roll))
	(terpri)
	(princ "Rolled:")
	(princ dices)
	(setq result (sum dices))
	
	(cond 
		((win_d result) Nil)
		((not (reroll_d result)) result)
		(T (turn))
	)
)

(defun second_turn ()
	(terpri)
	(princ "Second player's turn")
	(setq roll2 (turn))
	(terpri)
	(if (eval roll2)
		(cond
			((> roll1 roll2) (princ "First player wins!"))
			((= roll1 roll2) (princ "Draw!"))
			(T "Second player wins!"))
		(princ "Second player wins!")
	)
)

(defun game ()
	(terpri)
	(princ "First player's turn")
	(setq roll1 (turn))
	(terpri)
	(if (eval roll1)
		(second_turn)
		(princ "First player wins!")
	)
)
\end{lstlisting} 

\section{Написать функцию, которая по своему списку-аргументу lst определяет 
	является ли он палиндромом (то есть равны ли lst и (reverse lst)).}

\begin{lstlisting}
(defun is_palindrom (lst)
	(equalp lst (reverse lst))
)
\end{lstlisting} 


\section{Напишите свои необходимые функции, которые обрабатывают таблицу из
	4-х точечных пар:
	(страна . столица), и возвращают по стране - столицу, а по столице — 
	страну}

\begin{lstlisting}
(defun show_on_map (lst item)
	(cond
		((equal (caar lst) item) (cdar lst))
		((equal (cdar lst) item) (caar lst))
		((cdr lst) (show_on_map (cdr lst) item))
	)
)
\end{lstlisting}

\section{Напишите функцию, которая умножает на заданное число-аргумент 
первый числовой элемент списка из заданного 3-х элементного списка аргумента.}
Когда:
a) все элементы списка --- числа,
b) элементы списка -- любые объекты.

\begin{lstlisting}
(defun mul_a (lst num)
	(cond
		((and
			(numberp num)
			(and
				(numberp (car lst))
				(and
					(numberp (cadr lst))
					(numberp (caddr lst))
				)
			)
		) (* (car lst) num))
		(T Nil)
	)
)
\end{lstlisting}


\begin{lstlisting}
(defun mul_b (lst num)
	(cond
		((and (numberp (car lst)) (numberp num)) (* (car lst) num))
		((cdr lst) (mul_first_num (cdr lst) num))
		(T Nil)
	)
)
\end{lstlisting}


\end{document}