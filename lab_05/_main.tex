% !TeX spellcheck = ru_RU
\documentclass[a4paper, 14pt, unknownkeysallowed]{extreport}
\include{../common/settings}

\begin{document}
\include{title}
\setcounter{page}{2}

\chapter*{Введение}
\addcontentsline{toc}{chapter}{Введение}
Цель работы: приобрести навыки использования функционалов.

Задачи работы: изучить работу и методы использования применяющих и
отображающих функционалов: apply, funcall, mapcar, maplist.

\chapter{Практические задания}

\section{Напишите функцию, которая уменьшает на 10 все числа из списка-аргумента этой 
	функции, проходя по верхнему уровню списковых ячеек. ( * Список смешанный 
	структурированный)}
\lst{5}{12}{}{}

\section{Написать функцию которая получает как аргумент список чисел, а возвращает список 
	квадратов этих чисел в том же порядке.}

\lst{29}{35}{}{}

\section {Напишите функцию, которая умножает на заданное число-аргумент все числа из 
заданного списка-аргумента}
а) все элементы списка --- числа,
б) элементы списка -- любые объекты.

\lst{42}{48}{}{}
\lst{50}{59}{}{}

\section{Написать функцию, которая по своему списку-аргументу lst определяет является ли он
	палиндромом (то есть равны ли lst и (reverse lst)), для одноуровнего смешанного 
	списка.}

\clearpage
\lst{65}{78}{}{}

\section{Используя функционалы, написать предикат set-equal, который возвращает t, если два 
	его множества-аргумента (одноуровневые списки) содержат одни и те же элементы, 
	порядок которых не имеет значения.}
\lst{84}{94}{}{}

\section{Напишите функцию, select-between, которая из списка-аргумента, содержащего только
	числа, выбирает только те, которые расположены между двумя указанными числами - 
	границами-аргументами и возвращает их в виде списка.}

\lst{101}{110}{}{}

\section{Написать функцию, вычисляющую декартово произведение двух своих списков аргументов.}
Напомним, что А х В это множество всевозможных пар (a b), где а принадлежит А, принадлежит В.
\lst{115}{123}{}{}


\section{Почему так реализовано reduce, в чем причина?}

\lst{126}{127}{}{}

Если список пуст и не задано начальное значение, вызывается функция без аргументов. 

(+) -> 0

(*) -> 1

\section{ Пусть list-of-list список, состоящий из списков. Написать функцию, которая 
	вычисляет сумму длин всех элементов list-of-list (количество атомов)}
Например 
для аргумента
((1 2) (3 4)) -> 4

\lst{135}{144}{}{}


\end{document}