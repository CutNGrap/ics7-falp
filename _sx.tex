% !TeX spellcheck = ru_RU
\documentclass[a4paper, 14pt, unknownkeysallowed]{extreport}
\include{../common/settings}

\begin{document}
\include{title}
\setcounter{page}{2}

\chapter*{Введение}
\addcontentsline{toc}{chapter}{Введение}
Цель работы: приобрести навыки организации рекурсии в Lisp.

Задачи работы: изучить способы организации хвостовой, дополняемой, множественной,
взаимной рекурсии и рекурсии более высокого порядка в Lisp.

\chapter{Практические задания}

\section{ Написать хвостовую рекурсивную функцию my-reverse, которая развернет верхний 
	уровень своего списка-аргумента lst.}
\lst{2}{11}{}{}

\section{ Написать функцию, которая возвращает первый элемент списка - аргумента, который сам 
	является непустым списком.}

\lst{14}{19}{}{}

\section{Напишите рекурсивную функцию, которая умножает на заданное число-аргумент все 
	числа из заданного списка-аргумента}
a) все элементы списка --- числа,
6) элементы списка -- любые объекты.


\lst{30}{38}{}{}
\lst{39}{49}{}{}

\section{Напишите функцию, select-between, которая из списка-аргумента, содержащего только 
числа, выбирает только те, которые расположены между двумя указанными границами аргументами и возвращает их в виде списка }

\lst{57}{66}{}{}


\section{Написать рекурсивную версию (с именем rec-add) вычисления суммы чисел заданного списка:
	а) одноуровнего смешанного,
	б) структурированного.}

\lst{73}{82}{}{}

\lst{85}{94}{}{}


\end{document}