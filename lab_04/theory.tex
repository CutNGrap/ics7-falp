\chapter{Краткие теоретические сведения}
Функциональное программирование ориентировано на символьную обработку данных. Предполагается, что любую информацию можно свести к символьной. Слово <<символ>> здесь близко к понятию <<знак>>. Базис Lisp образуют: атомы, структуры, базовые функции, базовые функционалы.

\section{Данные}
Вся информация (данные и программы) в Lisp представляется в виде символьных выражений --- S-выражений. По определению

\begin{equation}
	S\text{-выражение} ::= \text{<атом>} | \text{<точечная пара>}.
\end{equation}

Атомы:
\begin{itemize}
	\item символы (идентификаторы) --- синтаксически --- набор литер (букв и цифр), начинающихся с буквы;
	\item специальные символы --- \{Т, Nil\} (используются для обозначения логических констант);
	\item самоопределимые атомы --- натуральные числа, дробные числа (например $\frac{2}{3}$), вещественные числа, строки – последовательность
	символов, заключенных в двойные апострофы (например “abc”);
\end{itemize}

Более сложные данные --- списки и точечные пары (структуры) строятся из
унифицированных структур --- блоков памяти --- бинарных узлов.
Определения:
\begin{align*}
	\text{<точечная пара>} &::= (\text{<атом>.<атом>}) | (\text{<атом>.<точечная пара>}) | \\
	&(\text{<точечная пара>.<атом>}) | \\
	&(\text{<точечная пара>.<точечная пара>}); \\
	\text{<список>} &::= \text{<пустой список>} | \text{<непустой список>} \\
	\text{<пустой список>} &::= ( ) | Nil \\
	\text{<непустой список>} &::= (\text{<первый элемент>.<хвост>}) \\
	\text{<первый элемент>} &::= \text{<S-выражение>} \\
	\text{<хвост>} &::= \text{<список>}
\end{align*}

Любая структура (точечная пара или список) заключается в круглые скобки:
(A.B) --- точечная пара, (А) - список из одного элемента. Пустой список изображается как Nil или ( ). Непустой список по определению может быть изображен: (A.(B.(C.(D())))). Допустимо изображение списка последовательностью атомов, разделенных пробелами: (A B C D). Элементы списка могут, в свою очередь, быть списками (любой список заключается в круглые скобки), например: (A (B C) (D (E))). Таким образом, синтаксически наличие скобок является признаком структуры --- списка или точечной пары. Любая непустая структура Lisp в памяти представляется списковой ячейкой, хранящей два указателя: на голову (первый элемент) и хвост --- все остальное.
