% !TeX spellcheck = ru_RU
\documentclass[a4paper, 14pt, unknownkeysallowed]{extreport}
\include{../../common/settings}

\begin{document}
\include{title}
\setcounter{page}{2}

\chapter*{Введение}
\addcontentsline{toc}{chapter}{Введение}
Цель работы: изучить структуру, особенности и принципы оформления
программы, способ выполнения программы на Prolog.

Задачи работы: приобрести навыки декларативного описания предметной области с
использованием фактов, правил и некоторых специальных разделов программы.
Изучить порядок использования фактов и правил в программе на Prolog, принципы и
особенности сопоставления и отождествления термов, на основе механизма унификации.

\chapter{Практические задания}
Создать базу знаний «Собственники», дополнив (и минимально изменив) базу
знаний, хранящую знания (лаб. 13), знаниями о дополнительной собственности владельца. Преобразовать знания об автомобиле к форме знаний о собственности. 
Вид собственности (кроме автомобиля):
\begin{itemize}
\item Строение, стоимость и другие его характеристики;
\item Участок, стоимость и другие его характеристики;
\item Водный транспорт, стоимость и другие его характеристики.
\end{itemize}
Описать и использовать вариантный домен: Собственность. Владелец может иметь, но 
только один объект каждого вида собственности (это касается и автомобиля), или не 
иметь некоторых видов собственности. 
Используя конъюнктивное правило и разные формы задания одного вопроса (пояснять 
для какого задания – какой вопрос), 
обеспечить возможность поиска:
\begin{enumerate}
\item Названий всех объектов собственности заданного субъекта,
\item Названий и стоимости всех объектов собственности заданного субъекта,
\item Разработать правило, позволяющее найти суммарную стоимость всех 
объектов собственности заданного субъекта.
\end{enumerate}
Для 2-го пункт и одной фамилии составить таблицу, отражающую конкретный 
порядок работы системы, с объяснениями порядка работы и особенностей использования 
доменов (указать конкретные Т1 и Т2 и полную подстановку на каждом шаге).

\clearpage
\section{Код программы}
\lst{1}{69}{lab}{Код программы}{../src/lab_08.pro}
\clearpage
\includepdf[pages=-]{table.pdf}


\end{document}