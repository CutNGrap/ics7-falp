% !TeX spellcheck = ru_RU
\documentclass[a4paper, 14pt, unknownkeysallowed]{extreport}
\include{../../common/settings}

\begin{document}
\include{title}
\setcounter{page}{2}

\section*{Введение}
\addcontentsline{toc}{chapter}{Введение}
Цель работы: изучить использование правил в программе: структуру,
особенности оформления, а также, способ и принципы выполнения таких программ на
Prolog.

Задачи работы: : приобрести навыки эффективного декларативного описания предметной
области с использованием фактов, правил и некоторых специальных разделов программы.
Изучить порядок использования фактов и правил в программе на Prolog, принципы и
особенности сопоставления и отождествления термов, на основе механизма унификации.
Способ формирования и изменения резольвенты. Порядок формирования ответа.

\chapter{Практические задания}

1. Создать базу знаний «Предки», позволяющую наиболее эффективным способом
(за меньшее количество шагов, что обеспечивается меньшим количеством
предложений БЗ - правил), и используя разные варианты (примеры) простого вопроса,
(указать: какой вопрос для какого варианта) определить:

1. по имени субъекта определить всех его бабушек (предки 2-го колена),

2. по имени субъекта определить всех его дедушек (предки 2-го колена),

3. по имени субъекта определить всех его бабушек и дедушек (предки 2-го
колена),

4. по имени субъекта определить его бабушку по материнской линии (предки 2-го
колена),

5. по имени субъекта определить его бабушку и дедушку по материнской линии
(предки 2-го колена).

Минимизировать количество правил и количество вариантов вопросов. Использовать
конъюнктивные правила и простой вопрос. Для одного из вариантов ВОПРОСА задания 1
составить таблицу, отражающую конкретный порядок работы системы.


2. Дополнить базу знаний правилами, позволяющими найти

1. Максимум из двух чисел

а) без использования отсечения,

в) с использованием отсечения;

2. Максимум из трех чисел

а) без использования отсечения,

в) с использованием отсечения;

Убедиться в правильности результатов.
Для каждого случая пункта 2 обосновать необходимость всех условий тела.
Для одного из вариантов ВОПРОСА и каждого варианта задания 2 составить
таблицу, отражающую конкретный порядок работы системы.

\clearpage
\section{Код программы}
\lst{1}{69}{lab}{Код программы}{../src/lab_08.pro}
\clearpage
\section{Таблицы порядка работы системы}
\includepdf[pages=-]{table1.pdf}
\includepdf[pages=-]{table2.pdf}


\end{document}