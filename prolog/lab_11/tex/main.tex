% !TeX spellcheck = ru_RU
\documentclass[a4paper, 14pt, unknownkeysallowed]{extreport}
\include{../../common/settings}

\begin{document}
\include{title}
\setcounter{page}{2}

\chapter{Практические задания}


\section*{Введение}
\addcontentsline{toc}{chapter}{Введение}
Цель работы:изучить рекурсивные способы организации программ на Prolog,
методы формирования эффективных рекурсивных программ и порядок реализации
таких программ.

Задачи работы: : приобрести навыки использования рекурсии на Prolog, эффективного
способа ее организации и прядка работы соответствующей программы.

\section*{Постановка задачи}
1. n!,
2. n-е число Фибоначчи.
Убедиться в правильности результатов.
Для одного из вариантов ВОПРОСА и каждого задания составить таблицу, отражающую
конкретный порядок работы системы:
Т.к. резольвента хранится в виде стека, то состояние резольвенты требуется отображать в
столбик: вершина – сверху! Новый шаг надо начинать с нового состояния резольвенты!
Для одного из вариантов ВОПРОСА составить
таблицу, отражающую конкретный порядок работы системы.

\clearpage
\section{Код программы}
\lst{1}{29}{lab}{Код программы}{../src/lab_08.pro}
\section{Таблицы порядка работы системы}
\includepdf[pages=-]{table1.pdf}
\includepdf[pages=-]{table2.pdf}


\end{document}