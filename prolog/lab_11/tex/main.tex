% !TeX spellcheck = ru_RU
\documentclass[a4paper, 14pt, unknownkeysallowed]{extreport}
\include{../../common/settings}

\begin{document}
\include{title}
\setcounter{page}{2}

\chapter{Практические задания}


\section*{Введение}
\addcontentsline{toc}{chapter}{Введение}
Цель работы:изучить рекурсивные способы организации программ на Prolog,
методы формирования эффективных рекурсивных программ обработки списков и
порядок их реализации.

Задачи работы: приобрести навыки использования списков на Prolog, эффективного способа
их обрвботки, организации и прядка работы соответствующих программ.
Изучить особенность использования переменных при обработке списков. Способ
формирования и изменения резольвенты в этом случае и порядок формирования ответа.

\section*{Постановка задачи}
Используя хвостовую рекурсию, разработать (комментируя назначение
аргументов) эффективную программу , позволяющую:

1. Найти длину списка (по верхнему уровню);

2. Найти сумму элементов числового списка;

3. Найти сумму элементов числового списка, стоящих на нечетных позициях исходного
списка (нумерация от 0);

4. Сформировать список из элементов числового списка, больших заданного значения;

5. Удалить заданный элемент из списка (один или все вхождения).

6. Объединить два списка.

Убедиться в правильности результатов.
Для одного из вариантов ВОПРОСА уметь составить таблицу, отражающую конкретный
порядок работы системы.

\clearpage
\section{Код программы}
\lst{1}{38}{lab}{Код программы}{../src/lab_08.pro}
\lst{40}{71}{lab}{Код программы (продолжение)}{../src/lab_08.pro}

\includepdf[pages=-]{table.pdf}

\end{document}