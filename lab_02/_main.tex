% !TeX spellcheck = ru_RU
\documentclass[a4paper, 14pt, unknownkeysallowed]{extreport}
\include{../common/settings}

\begin{document}
\include{title}
\setcounter{page}{2}

\chapter*{Введение}
\addcontentsline{toc}{chapter}{Введение}
Цель работы: приобрести навыки создания и использования функций пользователя в Lisp.

Задачи работы:
\begin{itemize}
	\item изучить работу интерпретатора Lisp;
	\item изучить алгоритм работы функции eval;
	\item изучить структуру и порядок обработки программы в Lisp.
\end{itemize}

\chapter{Практические задания}

\section{Составить диаграммы вычисления для \\выражений}

\img{1}{1_1}{}
\img{1}{1_2}{}
\img{1}{1_3}{}
\img{1}{1_4}{}
\img{1}{1_5}{}
\img{1}{1_6}{}

\clearpage
\section{Написать функцию, вычисляющую \\ гипотенузу прямоугольного треугольника по заданным катетам и составить диаграмму её вычисления}

\begin{lstlisting}
(defun hyp (a, b)
	(sqrt (+ (* a a) (* b b)))
)
\end{lstlisting}

\img{1}{2_1}{}

\section{Каковы результаты вычисления следующих выражений?}

\begin{lstlisting}
(list 'a c) ; unbound variable c, solution: (list 'a 'c)
(cons 'a (b c)) ; unbound variable (b c), solution: (cons 'a '(b c))
(caddr (1 2 3 4 5)) ; illegal function call, solution: (caddr '(1 2 3 4 5))
(cons 'a 'b 'c) ; invalid number of arguments: 3, solution:  (cons 'a (cons'b 'c))
(list 'a (b c)) ; unbound variable b, solution: (list 'a '(b c))
(list a '(b c)) ; unbound variable a, solution: (list 'a '(b c))
(list (+ 1 '(length '(1 2 3)))) ; type error, solution: (list (+ 1 (length '(1 2 3))))
\end{lstlisting}

\section{Написать функцию longer\_then от двух\\ списков-аргументов, которая возвращает Т, если первый аргумент имеет большую\\ длину}

\begin{lstlisting}
(defun longer_then (a b)
	(> (length a) (length b))
\end{lstlisting}

\section{Каковы результаты вычисления следующих выражений?}

\begin{lstlisting}
(cons 3 (list 5 6))	;(3 5 6)
(cons 3 '(list 5 6)) ;(3 list 5 6)
(list 3 'from 9 'lives (- 9 3))	;(3 from 9 lives 6)
(+ (length for 2 too)) (car '(21 22 23))) ;the variable FOR is unbound
(cdr '(cons is short for ans))	;(is short for ans)
(car (list one two))	;the variable ONE is unbound
(car (list 'one 'two)) ;one
\end{lstlisting}

\section{Какие результаты вычисления следующих выражений?}
Дана функция (defun mystery (x) (list (second x) (first x))).

\begin{lstlisting}
(mystery (one two)) ;undefined function ONE
(mystery (last one two)) ;undefined function LAST
(mystery free) ;the variable FREE is unbound
(mystery one 'two) ;the variable ONE is unbound
\end{lstlisting} 

\section{Написать функцию, которая переводит \\температуру в системе Фаренгейта в \\температуру по Цельсию.}

\begin{lstlisting}
(defun f-to-c (x) 
	(* (/ 5 9) (- x 32)))
\end{lstlisting} 

Как бы назывался роман Р.Брэдбери "+451 по Фаренгейту" в системе по Цельсию?
Ответ: 232.78 по Цельсию

\section{Что получится при вычисления каждого из выражений?}

\begin{lstlisting}
(list 'cons t NIL)	;(cons t Nil)
(eval (list 'cons t NIL)) ;(T)
(eval (eval (list 'cons t NIL))) ;function T is undefined
(apply #'cons '(t NIL))	;(T)
(eval NIL) ;Nil
(list 'eval NIL) ;(eval Nil)	
(eval (list 'eval NIL)) ;(Nil)
\end{lstlisting} 

\end{document}