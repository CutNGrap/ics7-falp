% Формат бумаги, шрифт
\documentclass[14pt, twoside, a4paper]{extreport}
\usepackage{extsizes}

% Русский язык
\usepackage[T2A]{fontenc}
\usepackage[utf8]{inputenc}
\usepackage[english,russian]{babel}

% Поля, абзацы
\usepackage[left=30mm,right=10mm,top=20mm,bottom=20mm]{geometry}

% Межстрочный интервал
\usepackage{setspace}
\onehalfspacing

% Абзацный отступ
\usepackage{indentfirst}
\setlength{\parindent}{1.25cm}

% Без отступа в списках
\usepackage{enumitem}
%\setlist{nolistsep}

% Тире вместо точки в списках
\renewcommand{\labelitemi}{---}
\renewcommand{\labelenumi}{\arabic{enumi})}

% Вставка графических изображений
\usepackage{graphicx}
\graphicspath{ {./img/} }

% Листинги кода
\usepackage{listings}
\lstset{
	basicstyle=\footnotesize\ttfamily,
	numbers=left,
	numberstyle=\tiny,
	numbersep=5pt,
	tabsize=4,
	breaklines=true,
	frame=b,
	stringstyle=\ttfamily\ttfamily,
	showspaces=false,
	showtabs=false,
	xleftmargin=17pt,
	framexleftmargin=17pt,
	framexrightmargin=5pt,
	framexbottommargin=4pt,
	showstringspaces=false,
	inputencoding=utf8x,
	keepspaces=true,
	numberbychapter=false
}

% Дополнительные возможности в формулах
\usepackage{amsmath}
\usepackage{amsfonts}
\usepackage{mathtools}

% Вставка графиков
\usepackage{tikz}
\usepackage{pgfplots}

% Подчёркивание всей области (\uline)
\usepackage[normalem]{ulem}

% Переопределение заголовков
\usepackage{titlesec}

%\AddToHook{cmd/section/before}{\clearpage}
\titleformat{\chapter}[block]{\bfseries\normalsize\filcenter}{\thechapter}{1em}{}
\titlespacing\chapter{\parindent}{\parskip}{1em}

\titleformat{\section}[hang]{\bfseries\normalsize}{\thesection}{1em}{}
\titlespacing\section{\parindent}{\parskip}{\parskip}

\titleformat{\subsection}[hang]{\bfseries\normalsize}{\thesubsection}{1em}{}
\titlespacing\subsection{\parindent}{\parskip}{\parskip}

% Диагональное разделение первой ячейки в таблицах
%\usepackage{slashbox}

% Правильные подписи
\usepackage{caption}
\captionsetup[figure]{justification=centering}
\DeclareCaptionLabelSeparator{emdash}{\ ---\ }
\captionsetup[figure]{name={Рисунок},labelsep=emdash}
\captionsetup[lstlisting]{labelsep=emdash}
\captionsetup[table]{singlelinecheck=false, labelsep=emdash}

% Сквозная нумерация таблиц и изображений
\counterwithout{figure}{chapter}
\counterwithout{table}{chapter}
\counterwithout{equation}{chapter}

% Дочерние уравнения
\newenvironment{sequations} {
\begin{subequations}
\renewcommand{\theequation}{\arabic{parentequation}.\arabic{equation}}
}{
\end{subequations}
}

% Подписи в титульном листе
\usepackage{array}
\newenvironment{signstabular}[1][1]{
	\renewcommand*{\arraystretch}{#1}
	\tabular
}{
	\endtabular
}

% Эта штука, каким-то образом, центрирует таблицы и изображение :/
\makeatletter
\g@addto@macro\@floatboxreset\centering
\makeatother

% Формат индексации источников в списке
\makeatletter
\renewcommand\@biblabel[1]{#1.}
\makeatother

% Изображения боком
\usepackage{rotating}

% Переименование списка иточников
\addto\captionsrussian{\renewcommand{\bibname}{Источники}}