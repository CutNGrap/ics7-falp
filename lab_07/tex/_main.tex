% !TeX spellcheck = ru_RU
\documentclass[a4paper, 14pt, unknownkeysallowed]{extreport}
\include{../../common/settings}

\begin{document}
\include{title}
\setcounter{page}{2}

\chapter*{Введение}
\addcontentsline{toc}{chapter}{Введение}
Цель работы: познакомиться со средой Visual Prolog, познакомиться со
структурой программы: способом запуска и формой вывода результатов.

Задачи работы: изучить принципы работы в среде VisualProlog, возможность получения
однократного и многократного результата, изучить базовые конструкции языка Prolog,
структуру програмым Prolog, форму ввода исходных данных и вывода результатов работы
программы.

\chapter{Практические задания}


\end{document}